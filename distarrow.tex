% I need better structure and better tikz skills
\documentclass{article}

\usepackage[vcentering]{geometry}
\geometry{papersize={6in,6.0in}, top=0.2in, bottom=0.2in}

\pagestyle{empty}

\usepackage{tikz}
\usetikzlibrary{
	arrows,decorations.pathmorphing,backgrounds,positioning,fit,calc,scopes
}

\tikzset{
	auto,
	compartment/.style={
		rectangle, minimum size=9mm, rounded corners=2mm,
		thick, draw=black!15, top color=white,bottom color=black!30
	},
	%
	bigcompartment/.style={
		rectangle, minimum size=24mm, rounded corners=2mm,
		thick, draw=black!15, top color=white,bottom color=black!20
	},
	%
	point/.style={
		circle, inner sep=2pt, fill=black!5
	},
	%
	mytextbox/.style={
		rectangle, text=black!50, thin, 
		draw=white, top color=white,bottom color=white, fill=white
	}
}

\begin{document}

\tikzstyle{line} = [draw, -latex']

\newcommand{\Poisson}{
	\node(P)[bigcompartment]{Poisson};
}


\newcommand{\Binomial}{
	\node(PB)[right=of P]{};
	\node(PBB)[right=of PB]{};
	\node(B)[bigcompartment, right=of PBB]{Binomial};
	\path[line] (P) -- node [text width=3cm, below, midway] {Finite pop} (B);
}

\newcommand{\Negative}{
	\node(PN)[below=of P]{};
	\node(PNN)[below=of PN]{};
	\node(N)[bigcompartment, below=of PNN]{Negative Binomial};
	\path[line] (P) -- node [text width=3cm, right, midway] {Pop-level variation} (N);
}

\newcommand{\Beta}{
	\node(BB)[below=of B]{};
	\node(BBB)[below=of BB]{};
	\node(Beta)[bigcompartment, below=of BBB]{Beta binomial};
	\path[line] (B) -- (Beta);
	\path[line] (N) -- (Beta);
}

\begin{tikzpicture}[font=\LARGE]
\Poisson
\end{tikzpicture}

\clearpage
\begin{tikzpicture}[font=\LARGE]
\Poisson
\Binomial
\end{tikzpicture}

\clearpage
\begin{tikzpicture}[font=\LARGE]
\Poisson
\Binomial
\Negative
\end{tikzpicture}

\clearpage
\begin{tikzpicture}[font=\LARGE]
\Poisson
\Binomial
\Negative
\Beta
\end{tikzpicture}

\end{document}

